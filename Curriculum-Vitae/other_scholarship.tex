% software macro
% #1: URL
% #2: Name
% #3: Description
\newcommand\software[4]{%
    #1 & \href{#2}{#3} & #4\\
}
\begin{tabularx}{\linewidth}{llX}
  \heading{Other Scholarship}
  \addlinespace
  %\subheading{Open-source Software}
  \multicolumn{3}{l}{\bfseries\sffamily Open-source Software}\\
  \multicolumn{3}{p{6.2in}}{%
        \itshape
        \begin{itemize}
            \item Open-source software repositories which I have spearheaded or
                made significant contributions to. The entries in the `Name'
                column are clickable links to the repositories.
            \item Repositories with an \inrank{} to their left correspond to
                repositories contributed to in rank.
            \iftoggle{apr}{%
                \item Repositories with an \annualreview{} to their left correspond
                    to repositories contributed to in the last three calendar years
                    (for the annual review process).
            }{}
        \end{itemize}
  }\\
  \addlinespace
  \addlinespace
  \midrule
  ID & Name & Description\\
  \midrule

  \software{\annualreview{}S10}{https://gitlab.com/artificialsocialintelligence/study4}{ASIST
  Study 4 Testbed}%
    {%
        The testbed used for ASIST Study 4. My contribution to the testbed was an
        updated version of the ASIST Study 3 Testbed's event extraction component
        that included a spellchecking system to meet the unique requirements of
        analyzing natural language in Study 4 (reproducibility,
        real-time output, high precision, and the ability to deal with
        domain-specific terms and new types of errors arising from the informal
        nature of text chat) that were not met by existing systems.
    }

  \software{\annualreview{}S9}{https://gitlab.com/artificialsocialintelligence/study3}{ASIST
  Study 3 Testbed}%
    {The testbed used for ASIST Study 3. My contributions to the testbed were
     components that performed real-time analysis (real-time transcription,
     event extraction, and labeling of sentiment/emotion) of multi-party
     spoken dialog in remote experiments.
    }

  \software{\inrank{}\annualreview{}S8}{https://github.com/ml4ai/skema}{SKEMA}%
    {Main repository for the SKEMA project, containing documentation and
    software for the text reading, structural alignment, and model role
    efforts.}

    \software{\inrank{}\annualreview{}S7}{https://github.com/ml4ai/tomcat}{ToMCAT}%
    {Main repository containing documentation and software for physio experiments.}

    \software{\annualreview{}S6}{https://github.com/ml4ai/tomcat-text}{ToMCAT DialogAgent}%
    {Real-time rule-based extraction of events from natural language.}

    \software{\annualreview{}S5}{https://github.com/ml4ai/tomcat-planrec}{ToMCAT plan recognition}%
    {Multi-agent plan recognition}

    \software{\annualreview{}S4}{https://github.com/ml4ai/tomcat-ASR\_Agent}%
    {ToMCAT ASR Agent}%
    {Real-time automatic speech recognition for multiple participants}

    \software{\annualreview{}S3}{https://github.com/ml4ai/tomcat-speechAnalyzer}%
    {ToMCAT SpeechAnalyzer}%
    {Real-time extraction of vocalic features, sentiment and emotion detection, and personality trait labeling.}

    \software{S2}{https://github.com/ml4ai/automates}%
    {AutoMATES}%
    {Automated Model Assembly from Text, Equations, and Software}

    \software{S1}{https://github.com/ml4ai/delphi}%
    {Delphi}%
    {Assembling causal, dynamic, probabilistic models from textual evidence and time series data.}


\end{tabularx}
