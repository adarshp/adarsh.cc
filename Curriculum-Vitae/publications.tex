\newpage
\newcommand{\publication}[2]{
    #1 & \fullcite{#2}\\\addlinespace\addlinespace
}

\newcommand{\tabitem}{~~\llap{\textbullet}~~}

\begin{ctabular}{rp{5.9in}}
    \heading{Publications/Creative Activity (Published or Accepted)}
    \multicolumn{2}{p{5.9in}}{%
        \begin{itemize}
            \item Co-authors who are student advisees or postdoctoral mentees
                have a $^\circ$ next to their names.
            \item Publications substantially based on work done as a
                graduate student have a * to their left.
            \item Items with an \inrank{} to their left correspond to activities
                performed in rank.
            \iftoggle{apr}{%
            \item Items with an \annualreview{} to their left correspond to
                entries in the last three calendar years (for the annual review
                process).
            }{}
        \end{itemize}
    }\\

    \subheading{Chapters in scholarly books and monographs}
    \publication{\annualreview{} B2.}{Zhang.ea:2022c}
    \publication{\annualreview{} B1.}{Pyarelal.ea:2022}

    \subheading{Refereed journal articles}
    \multicolumn{2}{p{6.3in}}{
            \emph{My journal articles are in the field of theoretical
                    particle physics, where it is conventional to order authors
                alphabetically by last name.}\newline
    }\\
    \entry{*J3.}{%
        \fullcite{Pyarelal:2020higgsino}
        \newline \emph{h5-index of Science China Physics, Mechanics \& Astronomy: 46 (as of 2024-02)}
    }\\
    \entry{*J2.}{%
        \fullcite{Kling2019}
        \newline \emph{Google Scholar Ranking:
        JHEP ranked \#2 in High Energy and Nuclear Physics as of 2024-02: (h5: 158)}
    }\\\addlinespace
    \entry{*J1.}{%
        \fullcite{Kling:2015uba}
        \newline \emph{Google Scholar Ranking:
        JHEP ranked \#2 in High Energy and Nuclear Physics as of 2024-02: (h5: 158)}
    }\\\addlinespace

    \subheading{Refereed conference articles}
    \multicolumn{2}{p{6.3in}}{\emph{In my primary fields of machine learning,
    artificial intelligence, and computational linguistics,
    conference publications are generally ranked higher than journal articles.
    These are full papers that go through the normal peer review process, as in
    a journal. In general, work that is published as a conference paper may not
    be submitted for publication elsewhere.
    Google Scholar Rankings and h5-indices are provided where available. Note
    that the acceptance rates, rankings and h5-indices are provided for the
    venue at the time of publication---thus, you may see different rankings and
    h5-indices for the same publication venue for different years.
    }}\\\addlinespace

    \entry{\inrank{}\annualreview{}C6.}{%
        \fullcite{Soares.ea:2024}
        %\newline \emph{Acceptance rate for the Datasets and Benchmarks Track of NeurIPS 2023: 32.7\%}
        \newline \emph{Google Scholar Ranking: ICML ranked \#3 in Artificial Intelligence (h5: 254)}
    }\\\addlinespace

    \entry{\inrank{}\annualreview{}C5.}
        \newline \emph{Google Scholar Ranking: NeurIPS ranked \#1 in Artificial Intelligence (h5: 309)}
    }\\\addlinespace

    \entry{\inrank{}\annualreview{}C4.}
        \newline \emph{Google Scholar Ranking: EMNLP ranked \#2 in Computational Linguistics (h5: 176)}
    }\\\addlinespace

    \entry{\inrank{}\annualreview{}C3.}
        \newline \emph{Google Scholar Ranking: EMNLP ranked \#2 in Computational Linguistics (h5: 176)}
    }\\\addlinespace

    \entry{C2.}
        \newline \emph{LREC ranked \#4 (by h5-index) in Computational Linguistics as of 2019}
    }\\\addlinespace

    \entry{C1.}{%
        \fullcite[][$^\dagger$ denotes equal contributions.]{sharp-etal-2019-eidos}
        \newline \emph{Google Scholar Ranking:
        NAACL-HLT ranked \#3 in Computational Linguistics as of 2024-02: (h5: 133)}
    }\\


    \subheading{Refereed workshop articles}
    \multicolumn{2}{p{6.3in}}{\emph{%
    Workshop publications are peer-reviewed publications, but less competitive
    than conference articles. They are meant for authors to get early feedback
    on their manuscripts prior to submitting them to competitive conferences.
    In general, workshops are `non-archival' venues, i.e., the research
    presented in a workshop paper can be submitted later to another venue
    (e.g., an `archival' venue such as a conference).}}\\\addlinespace

    \entry{\annualreview{}W8.}
        \newline \emph{Google Scholar Ranking:
        COLING ranked \#5 in Computational Linguistics as of 2024-02: (h5: 73)}
    }\\\addlinespace

    \entry{\annualreview{}*W7.}{%
        \fullcite{Kling:2022jcd}
    }\\\addlinespace

    \publication{\annualreview{}W6.}{Zhang.ea:2021}
    \publication{\annualreview{}W5.}{Soares.ea:2021}
    \publication{\annualreview{}W4.}{Pyarelal.ea:2021}
    \publication{W3.}{Pyarelal.ea:2019}
    \publication{W2.}{Pyarelal.ea:2019b}
    \publication{W1.}{Sharp.ea:2019b}

\end{ctabular}
